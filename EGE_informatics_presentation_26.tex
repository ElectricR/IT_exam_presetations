\documentclass{beamer}

\usepackage{listings}
\usepackage[utf8]{inputenc}
\usepackage[russian]{babel}

\usetheme{Copenhagen}
\usecolortheme{beaver}

\title[Задание 26]{ЕГЭ по Информатике 2021}
\subtitle{Задание 26}
\author{Роман Чулков}
\institute{СПбГУТ}

\begin{document}

\frame{\titlepage}

\begin{frame}

    \frametitle{Условие задачи}

    Системный администратор раз в неделю создаёт архив пользовательских файлов. Однако объём диска, куда он помещает архив, может быть меньше, чем суммарный объём архивируемых файлов. Известно, какой объём занимает файл каждого пользователя. 

    По заданной информации об объёме файлов пользователей и свободном объёме на архивном диске определите максимальное число пользователей, чьи файлы можно сохранить в архиве, а также максимальный размер имеющегося файла, который может быть сохранён в архиве, при условии, что сохранены файлы максимально возможного числа пользователей. 

\end{frame}

\begin{frame}

    \frametitle{Входные данные}

    В первой строке входного файла находятся два числа: S — размер свободного места на диске (натуральное число, не превышающее 10 000) и N — количество пользователей (натуральное число, не превышающее 3000). В следующих N строках находятся значения объёмов файлов каждого пользователя (все числа натуральные, не превышающие 100), каждое в отдельной строке.

\end{frame}

\begin{frame}

    \frametitle{Выходные данные}

    Запишите в ответе два числа: сначала наибольшее число пользователей, чьи файлы могут быть помещены в архив, затем максимальный размер имеющегося файла, который может быть сохранён в архиве, при условии, что сохранены файлы максимально возможного числа пользователей.
    
\end{frame}

\begin{frame}

    \frametitle{Пример}

    \begin{block}{Пример входного файла}

        100 4

        80

        30

        50

        40

    \end{block}

    При таких исходных данных можно сохранить файлы максимум двух пользователей. Возможные объёмы этих двух файлов \alert{30 и 40}, \alert{30 и 50} или \alert{40 и 50}. Наибольший объём файла из перечисленных пар — \alert{50}.

    \begin{alertblock}{Ответ}

        2 50

    \end{alertblock}

\end{frame}

\begin{frame}

    \frametitle{Решение}

    Воспользуемся \alert{жадной} стратегией для решения этой задачи.

    \pause

    \begin{alertblock}{Жадный шаг}

        Заметим, что файл с наименьшим размером всегда входит в оптимальное решение.

    \end{alertblock}

    \pause

    \begin{block}{Доказательство}

        Предположим, что $S_1$ - оптимальное заполнение архивного диска, которое не включает файл минимального размера. Заменим любой файл из $S_1$ на минимальный. Получим новое заполнение диска $S_2 \leq S_1$, а значит мы получили противоречие. 

    \end{block}

\end{frame}

\begin{frame}

    \frametitle{Алгоритм}

    Пусть $x_1, x_2, ... , x_N$ - множество файлов пользователей, и $S_0$ - изначальный объём архивного диска.

    Заметим, что сделав жадный шаг и добавив на архивный диск файл $x_i$, мы получим новую подзадачу, в которой множество файлов пользователей равно $\{x_1, x_2, ... , x_N\} \setminus x_i$ и объём архивного диска $S_1 = S_0 - x_i$. 

    \begin{alertblock}{Алгоритм}

        \begin{enumerate}

            \item Отсортируем файлы пользователей по возрастанию.

            \item Если множество файлов пусто, или объём архивного диска меньше минимального файла, закончим вычисления и выведем ответ.

            \item Изымем первый элемент массива файлов и добавим в ответ.

            \item Вычтем из объёма архивного диска вес текущего файла.

            \item Перейдем на пункт 2.

        \end{enumerate}

    \end{alertblock}

\end{frame}

\begin{frame}[fragile]

    \frametitle{Пример кода на Python}

    \begin{lstlisting}[language=Python, keywordstyle=\color{red}]

    i = 1

    print("asd")

    \end{lstlisting}

\end{frame}

\end{document}
